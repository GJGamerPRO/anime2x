\begin{onehalfspace}
	
\chapter{Kurzbeschreibung der Arbeit}

\section{Themeneinführung und Ausgangssituation}
Serien werden momentan immer populärer bei den jüngeren Generationen. Dazu gehören auch die aus Japan stammenden Anime, wobei es sich um animierte Zeichentrickversionen von Mangas handelt. Da die gängigen Distributionsmethoden solcher Serien eine Limitierung bei Auflösung und Bildwiederholrate erzwingen, hat ein Großteil der Serien dieses Genres, eine maximale Auflösung von 1920 · 1080 Pixeln bei einer Bildwiederholrate von 23,96 Hz. Werden die Serien über das Fernsehen ausgestrahlt, liegt die Auflösung meist nur bei 1280 · 720 Pixeln und bei Videomaterial, welches sich auf einer DVD befindet, ist die Auflösung sogar noch geringer. Da diese zum noch zum größeren Teil von Hand animiert werden, ist die Bildwiederholrate gering.  Gründe für diese geringen Werte sind wie bereits erwähnt Limitierungen der einzelnen Medien, jedoch auch terminliche und monetäre Restriktionen, an die die Animateure gebunden sind. Ziel der Arbeit ist es, ein Tool zu erschaffen, mit dem jeder, hochgeladenes Videomaterial einfach und ohne Probleme hochskalieren kann. Dabei fokussiert sich diese Arbeit ausschließlich auf das Hochskalieren der Auflösung z. B. 720p → 1080p, 1080p → 2k. 

Es existieren bereits einige Werkzeuge, mit denen es möglich ist, einzelne Bilder hochzuskalieren. Die unterschiedlichen Upscaler unterscheiden sich jedoch in ihren Fähigkeiten, verschiedene Arten an Bildern hochzuskalieren. Es existieren welche, die sich für realistische Bilder eignen und andere, die sich auf gezeichnete Inhalte, wie Animes, spezialisieren. Die meisten Anime-Upscalern sind zum frei zum Download verfügbar und müssen somit auf den eigenen Computern installiert und ausgeführt werden. Die meisten Upscaler verwenden künstliche Intelligenz und basieren oft auf Frameworks wie PyTorch und TensorFlow, welche oft die CUDA-Technologie voraussetzen, um mit einer annehmbaren Geschwindigkeit arbeiten zu können. Zudem sind die meisten Upscaler reine Bild-Upscaler und demnach können diese Videos nicht direkt hochskalieren. Für zum Hochskalieren von Videos im Anime-Zeichenstil wird bisher auch kein Service angeboten, den man nicht bei sich selbst installieren muss.

\section{Zielsetzung}

Die Probleme, die sich aus der aktuellen Situation ergeben, sind unter anderem, dass Anime-Upscaler vielen Fans dieses Genres nicht zugänglich sind, da die Hürde durch das Installieren teilweise sehr hoch ist, einige Hardwareanforderungen aktuell äußerst hoch sind und bestimmte viele Anwendungen und Dienste nur Bilder hochskalieren können. Diese Problemstellungen sollen alle mit Hilfe eines Tools, welches webbasiert arbeitet, angegangen werden. Nutzer sollen ihr Videomaterial auf einer Website hochladen und dieses nach einer kurzen Wartezeit hochskaliert wieder herunterladen können. Je nach Verlauf der Arbeit ist zusätzlich die Erstellung einer lokalen Anwendung, welche sich einfach installieren lässt, auch ein Ziel, welches Teile der oben genannten Probleme angeht.

Ziel dieser Studienarbeit soll es daher sein, es einem Benutzer ohne Fachkenntnisse möglich ist einen Anime hochzuladen und diesen hochskaliert wieder herunterzuladen. Dies soll als „Software as a Service“ (SaaS) in einer Website umgesetzt werden. Damit benötigt der Nutzer lediglich Hardware, die es ermöglicht, auf die Website zuzugreifen und die Animes hochzuladen. Abhängigkeiten wie die CUDA-Technologie wären damit kein Benutzerproblem. Die Webseite sollte eine Benutzerkontenverwaltung enthalten, sodass jeder Benutzer nur auf sein eigenes Videomaterial Zugriff hat. Zudem sollte sie benutzerfreundlich gestaltet sein, sodass diese intuitive genutzt werden kann. Als eine Nebenbetrachtung sollen des Weiteren die rechtlichen Aspekte eines solchen Dienstes betrachtet werden, um Anforderungen der DSGVO und des TMG einzuhalten und straf- und zivilrechtlich abgesichert zu sein

\section{Ähnliche Arbeiten im Themengebiet}

Über die Zeit wurden verschiedene Ansätze zum Hochskalieren von Bildern entwickelt. Eine Möglichkeit wurde von Gilad Freedman und Raanan Fattal erforscht, die im Paper „Image and Video Upscaling from Local Self-Examples“ veröffentlicht wurde. Hierbei wird eine Technik angewendet, die mit Hilfe von lokalen Ähnlichkeiten die Auflösung erhöht. Dazu verwendet das Bild sich selbst als Referenz und vergleicht auf sehr kleiner Ebene Merkmale. Falls diese Merkmale die gleichen Muster aufweisen, werden diese auf eine höhere Auflösung angewendet. In ihrer Arbeit bezogen sich Herr Freedman und Fattal jedoch nur auf Naturbilder, da diese viele wiederholende Muster aufweisen. Abseits dieses und weiterer Beispiele wird Künstliche Intelligenz immer relevanter in dem Themenfeld der Bildverarbeitung. Vor einigen Jahren war diese Option aufgrund von fehlender Rechenleistung nicht gegeben, weswegen andere Methoden wie die Local Self-Examples aushelfen mussten. Heutzutage ist die benötigte Rechenleistung jedoch vorhanden, was das Hochskalieren von Bildern mit Hilfe von Künstlicher Intelligenz ermöglicht. Wie in „Image Super-Resolution Using Deep Convolutional Networks“ beschrieben, wird ein Convolutional Neural Network verwendet. Die letzten Jahre haben sich CNN als besonders potent im Bereich der Bilderkennung oder -bearbeitung herausgestellt. Diese Netze analysieren die Umgebung eines Pixels, um so das Bild mit der neuen Auflösung erstellen zu können. Für die Analyse gibt es verschiedene Ansätze, damit es für das menschliche Auge natürlich wirkt und das Ergebnis zufriedenstellend ist.

Für diese Studienarbeit sollte nach heutiger State-of-the-Art Technologie ein neuronales Netz verwendet werden, um die Animes auf eine höhere Auflösung zu bringen. Dabei muss analysiert werden, mit Hilfe welcher Konfigurationen Animes am besten hochskaliert werden können und wie performant die Lösungen sind. Der Punkt Performance wird in dieser Studienarbeit besonders relevant, da nicht nur einzelne Bilder hochskaliert werden sollen, sondern jedes Frame eines Videos. Bei einer durchschnittlichen Bildwiederholrate von 23,96 Bildern pro Sekunde, wird rasant viel Rechenleistung benötigt.



\end{onehalfspace}