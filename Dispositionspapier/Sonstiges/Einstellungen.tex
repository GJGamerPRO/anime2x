\usepackage[utf8]{inputenc}
\usepackage[ngerman]{babel} %sorgt auch für Inhaltsverzeichnis Deutsch/Englisch
\usepackage[top=25mm,left=30mm,right=30mm,bottom=25mm]{geometry}
\usepackage[backend=biber,style=numeric,citestyle=numeric,sorting=none]{biblatex}
\usepackage{csquotes}
\usepackage{hyperref}

%\usepackage[compact]{titlesec}
%\titlespacing{\chapter}{0pt}{*0}{*0}
%\titlespacing{\section}{0pt}{*0}{*0}
%\titlespacing{\subsection}{0pt}{*0}{*0}
%\titlespacing{\subsubsection}{0pt}{*0}{*0} 

\newcommand{\eigenname}[1]{\emph{#1}}
\usepackage{titlesec}
\usepackage[nonumberlist,acronyms]{glossaries}
\usepackage[printonlyused]{acronym}
\usepackage{amsmath}
\usepackage{siunitx}
\usepackage{setspace}
\usepackage{longtable}
\usepackage{calc}
\usepackage{graphicx}
\usepackage{appendix}
\usepackage[nottoc]{tocbibind}
\usepackage{nameref}
\usepackage{makecell}
\renewcommand\theadfont{\bfseries}
\usepackage{pdflscape}
\usepackage{pdfpages}
\usepackage{todonotes}

\usepackage[official]{eurosym}
\graphicspath{ {./Bilder/} }
%%%%%%%%%%%%%%%%%%%%%%%%%%%%%%%%%%%%%%%%%%%%%%%%%%%%
%KOPF UND FUSSZEILE
\usepackage[headsepline,footsepline,plainfootsepline,plainheadsepline]{scrlayer-scrpage}
\pagestyle{scrheadings}
\clearscrheadfoot
\cfoot[\newline\pagemark]{\newline\pagemark}




%%%%%%%%%%%%%%%%%%%%%%%%%%%%%%%%%%%%%%%%%%%%%%%%%%%
\makeglossaries
\glstoctrue
\sloppy

\titleformat{\chapter}[display]
{\normalfont\huge\bfseries}{\chaptertitlename\ \thechapter}{20pt}{\Huge}
\titlespacing*{\chapter}{0pt}{0pt}{40pt}


\addbibresource{Bib/bib.bib}

%########################################
%KATEGORIEN DER LITERATUR
%NEUE KATEGORIE
%\DeclareBibliographyCategory{Inet} 
%RESOURCE IN KATEGORIE EINFÜGEN
%\addtocategory{Inet}{stackoverflow2018survey}
%########################################

\DeclareBibliographyCategory{Inet}
\addtocategory{Inet}{stackoverflow2018survey}

%\geometry{top=25mm,left=25mm}



\sloppy

\usepackage{caption}
\newcommand{\source}[1]{\caption*{Source: {#1}} }

\usepackage{xparse}
\DeclareDocumentCommand{\newdualentry}{ O{} O{} m m m m } {
	\newglossaryentry{gls-#3}{name={#5},text={#5\glsadd{#3}},
		description={#6},#1
	}
	\makeglossaries
	\newacronym[see={[Glossary:]{gls-#3}},#2]{#3}{#4}{#5\glsadd{gls-#3}}
}
